\documentclass[../Matt_Gebert_Honours_Thesis.tex]{subfiles}

\begin{document}
	
%	I will then describe the devices and measurements I have made in \cref{chap:fab&characterisation}. This will regard connections to devices, which allow the measurements I have perform, and the processes used to fabricate our devices. I have made graphene devices using lithography and evaporation methods, to create electrical contacts. I will also describe the oxides I have investigated in this chapter, and the methods I have used to transfer them.

	Two aspects are involved in performing electronic measurements of materials, namely fabrication and measurement techniques. 
	
	Physical devices containing graphene need to be fabricated to allow measurements to take place. This primarily deals with the production of materials (graphene, oxides) and the processes to develop electrically connectable devices. Some of the tools and techniques used include lithography, electron beam evaporation, and etching. The processes used in this project are detailed in \cref{sec:fabrication}.
	
	To measure the electronic transport properties of graphene, particular experimental methods and procedures are required to control the environment and obtain useful data. \Cref{sec:measurements} primarily deals with the operation of the experimental apparatus used to measure the electronic properties of graphene.

	\section{Fabrication}\label{sec:fabrication}
	\subfile{sections/chap2/chap2_sec1_fabrication.tex}
		
	\section{Measurement techniques}\label{sec:measurements}
	\subfile{sections/chap2/chap2_sec2_measurement_techniques.tex}
	
\end{document}