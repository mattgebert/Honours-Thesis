\documentclass[../Matt_Gebert_Honours_Thesis.tex]{subfiles}

\begin{document}
% I describe the various ways of producing and identifying graphene in lab use, and the characterisations I have conducted. This will include our use of atomic force microscopy (AFM), optical microscopy and Raman spectroscopy.

\section{Production}
Since graphene's realisation in 2004 \cite{novoselov_electric_2004}, much research has been focused to finding effecient ways of producing large amounts of graphene \cite{zhang_review_2013}. Originally, the first samples ever created which have primarily been used for sensative measuremnts have been conducted using a method of exfoliation (\cref{sec:exfoliation}). These samples typically exhibit better electronic properties than those produced by other methods.
Since 2008/2009, CVD (\cref{sec:CVD}) of carbon to create graphene films has provided another prominent method to produce large films for industrial scale applications. In particular, growth of graphene on copper sheets \cite{li_large-area_2009} has been a reliable way producing these large uniform sheets.

There are other methods not used in this thesis, such as epitaxial growth of graphene via SiC uses heating to boil off silicon atoms to form a layer of graphene on it's surface.

\subsection{Exfolation}\label{sec:exfoliation}
Originally made famous in the breakthrough method by Giem and Novoselov \cite{novoselov_electric_2004, novoselov_two-dimensional_2005}, a mechanical exfoliation technique allowed for the isolation of atomically thin crystals of various materials. They reported the use of scotch tape to cleave thin layers from a larger crystal. 

The common procedure involves pressing tape/surfaces against a bulk crystal (such as highly orientated pyrolitic graphite (HOPG), Kish graphite, natural graphite, or graphenium). Due to van der Waals interactions, layers of graphite are transferred to the desired surface. By repeated peeling of the same tape, a thin coverage can be obtained and then transferred onto substrates, such as \silicondioxide.

\subsubsection{Thermal enhancement}
Drawing on the methods described in Huang \etal\cite{huang_reliable_2015}, we developed a reliable method of exfoliation. 
Using 

When bringing the tape with graphite flakes into contact with the \silicondioxide wafer, we use Huang \etal's method  they  use an annealing process of heating the tape and wafer for 2-5m at $\sim 100^\circ$C on a conventional lab hot plate. After allowing cooling to room temperature, the tape is removed. They find under optical microscopy that graphene flakes with uniform thickness routinely range from $\sim 20\mu$m to above $100\mu$m. The two additional steps to regular exfolation methods were oxygen plasma cleaning and temperature annealing. Annealing is expected to increase traction due to the remove of gas moelcules trapped between SiO$_2$ and graphite. Oxygen plasma is expected to remove absorbates on the substrate surface. 

Huang \textit{et al} suggest part of their success comes from applying larger uniform coverage on tape. They suggest that because the exfoliation comes from the competition of forces between the substrate and the other graphene layers in multilayer graphite, using thinly covered tape is detrimental to the transfer.



\paragraph{Extra details} Huang \etal also specified further details about the parameter spaces they search for optimising their exfoliation. 
\begin{itemize}
	\itemsep0em 
	\item Only apply tape exfoliation maximum 3 to 4 times after removal from bulk graphite.
	\item Annealing time nor temperature not strongly affected coverage, but optimal at $\sim100^\circ$C \& 2 mins. Longer time also implied more glue residue from the tape.
\end{itemize}

\subsubsection{Rough surface adhesion}


\subsection{CVD}\label{sec:CVD}

\subsubsection{Dry transfer}


\section{Indentification of Graphene}
\subsection{Optical Microscopy}

\subsection{Raman Spectroscopy}

\subsection{Atomic Force Microscopy Imaging}



\end{document}