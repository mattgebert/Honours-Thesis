\documentclass[../Matt_Gebert_Honours_Thesis.tex]{subfiles}

\begin{document}
% I describe the various ways of producing and identifying graphene in lab use, and the characterisations I have conducted. This will include our use of atomic force microscopy (AFM), optical microscopy and Raman spectroscopy.

\section{Production}

Since graphene's realisation in 2004 \cite{novoselov_electric_2004}, much research has been focused to finding effecient ways of producing large amounts of graphene \cite{zhang_review_2013}. Originally, the first samples ever created which have primarily been used for sensative measuremnts have been conducted using a method of exfoliation (\cref{sec:exfoliation}). These samples typically exhibit better electronic properties than those produced by other methods.
Since 2008/2009, CVD (\cref{sec:CVD}) of carbon to create graphene films has provided another prominent method to produce large films for industrial scale applications. In particular, growth of graphene on copper sheets \cite{li_large-area_2009} has been a reliable way producing these large uniform sheets.

There are other methods not used in this thesis. Epitaxial growth of graphene via SiC uses heating to boil off silicon atoms to form a layer of graphene on it's surface.
Chemical exfoliation 

\subsection{Exfolation}\label{sec:exfoliation}
\subsection{CVD}\label{sec:CVD}

\section{Indentification of Graphene}
\subsection{Optical Microscopy}

\subsection{Raman Spectroscopy}

\subsection{Atomic Force Microscopy Imaging}



\end{document}