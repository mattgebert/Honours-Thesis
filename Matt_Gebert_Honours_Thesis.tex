\documentclass{report}

%-----------------------PACKAGES-----------------------
\usepackage{geometry} %Allows changes of margins to document.
\usepackage{amsmath} %Allows text in equations
\usepackage{float} %Allows Placement of Images & Tables within pages, not at top etc. ie {table}[H]
%\usepackage{cite} %You want to refer to a bibtex document? But not needed with natbib.
\usepackage[super,square]{natbib}%Forces small number citations superscripted.
%\usepackage[natbibapa]{apacite} For particular citing styles?
\usepackage[dvipsnames]{xcolor} %Allows custom names and colour designs
%\usepackage{mathtools}
%\usepackage{tabularx} %Allows table widths to be set.
\usepackage[final]{pdfpages} %to insert pdf pages of code.
\usepackage{hyperref} %Hyperlinking \href{URL}{text}
\usepackage{cleveref} %Allows text like "Figure" to be automatically generated. "\cref{}" == "Tbl. \ref{}"
\usepackage{matlab-prettifier} %Matlab Code Inserts
\usepackage{braket} %Dirac Notations
\usepackage{rotating} %Allows rotation of an image 
\usepackage{tabularx} %Allows tables that have multiple lines per cell. https://tex.stackexchange.com/questions/194737/multiline-in-a-table-cell
\usepackage{dsfont} %Allows symbols like \mathds{1} for identity matrix. 
\usepackage{caption} %Allows use of subfigure environment with captions inside figure.
\usepackage{subcaption} %The same as above. https://tex.stackexchange.com/questions/37581
\usepackage{listings}
\usepackage{amssymb} %Allows \therefore.
\usepackage{wrapfig} %Allows wrapping of text around figures. \begin{wrapfigure}[lineheight]{position}{width}
\usepackage{enumitem} %Allows change of horizontal seperation distances \begin{itemize}[noitemsep,topsep=0pt]
\usepackage{graphicx} %Enables use of \DeclareGraphicsExtensions command and setting graphics path.

%\usepackage{glossaries} %To make a glossary.  \makeglossaries  \newglossaryentry{latex}{name= , description={}}
%						\newacronym{gcd}{GCD}{Greatest Common Divisor}
\usepackage[nottoc,numbib]{tocbibind} %Adds bibliography to TOC.
\usepackage{subfiles} %Allows a TEX document to be contained within multiple TEX Files.

%-----------------------COMMANDS-----------------------
%Mimmic Mathematica Behaviour.
\newcommand{\pd}{\partial}
%Code to make a glossary entry:
\newcommand{\gloss}[2]{\paragraph{}\textbf{#1}	\textit{#2}}
%Code to change colour to blue of a cite.
\newcommand{\cc}{\hypersetup{citecolor=blue}}
%Code to add single equation lines to align*'s
\newcommand\numberthis{\addtocounter{equation}{1}\tag{\theequation}}
%Defines a typeset for generating 'code' text.
\definecolor{lightgrey}{RGB}{200,200,200}
\newcommand{\code}[1]{\colorbox{lightgray}{\lstinline[basicstyle=\ttfamily\color{black}]|#1|}}
%Code to add \textit{et al} text.
\newcommand{\etal}{\textit{et al }}
\newcommand{\etals}{\textit{et al's}}
%Tensor product symbol
\newcommand{\tens}{\mathbin{\mathop{\otimes}}} 
%Trace text in equations:
\newcommand{\tr}[1]{\text{Tr}\left[#1\right]}
%Greater and Less than Similar commands simplified:
\newcommand{\gsim}{\gtrsim}
\newcommand{\lsim}{\lesssim}

%-----------------------DOCUMENT SETUP-----------------------
%Allows display breaks in equations across pages. More condensed.
\allowdisplaybreaks
%Change the bibliography name to references:
\renewcommand\bibname{References}
%Allows custom setup of URL link colours:
\hypersetup{
	linkcolor={red!50!black},
	citecolor={blue!50!black},
	urlcolor={blue!80!black},
	colorlinks = true,
	hidelinks=true
%	pdfborder={0 0 1}
}
\AtBeginDocument{\hypersetup{ 
			pdfborder={10 10 0.1 [1 0]}, %Required to go within "\AtBeginDocument" here otherwise overrided. {HorizontalCornerRadius VerticleCordnerRadius Thickness [Dashes PerUnits] }
			citebordercolor={0 0 1}, %{R G B} on a scale 0 to 1.
			linkbordercolor={0 0.9 0}, %{R G B} on a scale 0 to 1.	
		}}
%add specific resources
\DeclareGraphicsExtensions{.pdf,.png,.jpg,.svg}
\graphicspath{ {./graphics/} }

% --------------------- Document ----------------------------

\font\myTitle=cmr12 at 29pt
\title{}
\date{\today}
\author{Matthew Gebert}

\begin{document}
	\newgeometry{left=3cm, bottom=3cm}
	
	\begin{titlepage}
%		\maketitle
		\centering
		\scshape\LARGE Monash University\\
		\vspace{5mm}\Large Honours Thesis\\\vspace{1.5cm}
		
		\myTitle Thin oxides in graphene devices\\
		\normalfont\normalsize
		\vspace{5mm}\includegraphics[width=0.5\textwidth]{placeholder.png}
		\vspace{5mm}
		\begin{flushleft}
			\hrulefill
		\end{flushleft}
		\begin{tabularx}{\textwidth}{Xl}
			Supervisors:&Michael Fuhrer\\
			&Semonti Bhattacharyya
		\end{tabularx}\newline
		\begin{flushleft}
			\hrulefill
		\end{flushleft}
		\begin{center}
			Physics Honours, Monash University\\Student ID: 24121843
		\end{center}
		\vspace{3mm}
		
		\begin{center}
			\large\textbf{Abstract:}
		\end{center}
		
		I present a review of the use of graphene in electronic devices, both in its shortfalls and exciting properties. The electronic structure is detailed, along with various scattering sources that affect electron transport and ultimately the goal of room temperature, electronic devices. Considering heterostructures and the use of other materials to enhance graphene, I discuss the potential use of hafnium dioxide, and other oxides, as an excellent gate dielectric material for potential use in graphene field-effect devices.
	
	\end{titlepage}
	
	\newpage 
	\renewcommand{\baselinestretch}{0.94}\normalsize
	\tableofcontents
	\renewcommand{\baselinestretch}{1}\normalsize
	
	\section{Foreword}
	This thesis serves the purpose presenting the conclusions of my research into thin oxides on graphene. I will be arguing why I have come to the conclusions I have, and how that fits into a bigger picture of materials science and particular applications.
	
	In \cref{chap:introduction}, I will outline what I hope to achieve in this project. I begin by discussing the theoretical properties of graphene and why it has attracted so much interest as an electronic material. I will also describe some challenges facing new computing technologies, including the use of dielectrics, and how my work contributes to realising solutions to new generations of this technology. I will outline a theoretical and experimental summary of the results to date seen in introducing dielectrics to graphene.
	
	In \cref{chap:production&identification}, I describe the various ways of producing and identifying graphene in lab use, and the characterisations I have conducted. This will include our use of atomic force microscopy (AFM), optical microscopy and Raman spectroscopy.
	
	I will then describe the devices and measurements I have made in \cref{chap:characterisation}. This will particularly regard geometry and connections to devices,  which allow the measurements I have perform. 
	
	This will motivate my description of the processes used to fabricate our devices in \ref{chap:fabrication}. I have made graphene devices using lithography and evaporation methods, to create electrical contacts. I will also describe the oxides I have investigated in this chapter, and the methods I have used to transfer them.
	
	In \cref{chap:cvd} and \cref{chap:exfoliated}, I will present the data and results from my measurements of the respective devices, analysing the effects of oxides before and after stamping.
	
	\chapter{Introduction}\label{chap:introduction}
	\subfile{sections/introduction.tex}
	
	\chapter{Production \& identification of graphene}\label{chap:production&identification}
	\subfile{sections/chap1_graphene_finding.tex}
	
	\chapter{Device characterisation}\label{chap:characterisation}
	\subfile{sections/chap2_graphene_characterisation.tex}
	
	\chapter{Device fabrication techniques}\label{chap:fabrication}
	\subfile{sections/chap2_graphene_characterisation.tex}
	
	\chapter{CVD graphene}\label{chap:cvd}
	\subfile{sections/chap2_graphene_characterisation.tex}
	
	\chapter{Exfoliated graphene}\label{chap:exfoliated}
	\subfile{sections/introduction.tex}
	
	
\bibliographystyle{unsrt}
\bibliography{PhysicsHonours,External}

\end{document}